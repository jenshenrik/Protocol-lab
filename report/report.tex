\documentclass{article}

\usepackage[utf8]{inputenc}
\usepackage{lipsum}
\usepackage{listings}
\usepackage{mdwlist}

\newcommand\Q[1]{
	\leavevmode\par
	\noindent
	\emph{Q -- #1}
	\\
}

\newcommand\A[1]{
	A -- #1
}

\title{Protocol Security Lab}
\author{Jens Henrik Skuldbøl, s070145}
\date{October 25th, 2013}

\begin{document}

\maketitle

\section{Kerberos PKInit}

\subsection{Describe the protocol}
This protocol required the client to comunicate with two servers,
the authentication server, \texttt{ath}, and the ticket issuing server, 
\texttt{g}, before gaining access to the requested ressource.
First, the client contacts the authentication server, requesting access
to the desired ressource.
Assuming that the clients identity is verified, the server replies
with a session key, and a key the client shares with \texttt{g}.
This key is encrypted using a key unknown to the client,
a key shared between \texttt{ath} and \texttt{g},
which prevents it being tampered with.

The client then passes this session key on to \texttt{g},
along with his identity, encrypted with his and \texttt{g}'s shared key.
Assuming the client is authorized to access the requested ressource,
\texttt{g} issues and returns a ticket to the client.
This ticket is also encrypted with a key unknown to the client,
a key shared between \texttt{g} and the host of the ressource, \texttt{s},
to prevent tampering.

The client can now send this ticket to the server he wishes to access,
which then replies with the requested ressource.

\Q{Which of the messages can the client \emph{C} decrypt?}
\A{
	Strictly speaking, the client can decrypt all of the messages.
	He cannot, however, decrypt all parts of some of them, these begin
	particularly the session key, sent to him from \texttt{ath} encrypted
	using \texttt{skag}, the shared key between \texttt{ath} and \texttt{g}.
	Also, he cannot decrypt the ticked, sent to him from \texttt{g} encrypted
	using \texttt{skgs}, the shared key between \texttt{g} and \texttt{s}.
}

\Q{Which keys does \emph{C} ever learn?}
\A{
	From the beginning, the client knows the following keys
	\begin{itemize*}
		\item \texttt{pk(C)}/\texttt{inv(pk)c))} -- his own public/private keypair
		\item \texttt{pk(ath)} -- the public key of \texttt{ath}
	\end{itemize*}
	
	Communicating with \texttt{ath}, the client learns the following keys
	\begin{itemize*}
		\item \texttt{Ktemp} -- A temporary key used by \texttt{ath} to encrypt \texttt{KCG}
		\item \texttt{KCG} -- The shared key between the client and \texttt{g}
	\end{itemize*}

	Communicating with \texttt{g}, the client learns \texttt{KCS}, the key shared
	between him and \texttt{s}, the server he wishes to access.
}

\Q{How does the protocol prevent illegitimate access?}
\A{
	The protocol prevents illigitimate access by verifying both the users
	identity, and that he is authorized to access the requested ressource.
	The integrety of this authentication is kept by giving the ticket an
	expiration date, and by keeping both session keys and tickets encrypted
	between the authenticating servers, thus preventing them from being
	tampered with by anyone, including the user.
}


\subsection{Explain the attack}
The output of OFMC after finding the attack is included in \ref{app:pktrace}.
As can be seen from the first two messages of the attack, found in lines 17 and 18,
the intruder performs a 'man in the middle'-attack between the client and
the authentication server.
The intruder then subsequently exchanges the identity of the servers with his own,
making sure \texttt{C} keeps sending messages to him.

The protocol is fixed by encrypting the first message from the \texttt{C} to \texttt{ath}
with \texttt{ath}'s public key, thereby making it decipherable by \texttt{ath} only.

The fix is verified for two sessions using OFMC, with the output listed in \ref{app:pkfix-log}.
The first message then becomes
\begin{verbatim}
C -> ath: {C,g,N1,{T0,N1,hash(C,g,N1)}inv(pk(C))}pk(ath)
\end{verbatim}

\subsection{Design a variant}
The code for the variant is included in \ref{app:pkdh}.
Note that this variant builds on the changes made in the fix listed above.
First of all, there are some changes to the types and knowledge.
Added to the types are three numbers, \texttt{p}, \texttt{x} and \texttt{y},
and a function, \texttt{exp}.

\texttt{p} is the public number, known by both \texttt{C} and \texttt{ath}.
\texttt{x} and \texttt{y} are the secrets of \texttt{C} and \texttt{ath} respectively,
and the exponential function, \texttt{exp} is known by both of them.

Changes in communication are in the first message from \texttt{C} to \texttt{ath},
and the reply.
First of all, \texttt{C} includes in his message the calculation \texttt{exp(p,x)},
which is his half of the Diffie-Hellman key.
In its reply, \texttt{ath} now includes \texttt{exp(p,y)},
but instead of encrypting \texttt{KCG} with a temporary key,
it now encrypts it with the complete Diffie-Hellman key,
\texttt{enc(enc(p,x),y)}.

The verifying output of OFMC is listed in \ref{app:pkdh-fix}.

\section{AMP}

\subsection{Describe and explain the protocol}

\Q{What security relationship does the parties have initially?}
\A{
	Looking at the initial knowledge, \texttt{C} knows the public keys
	of \texttt{s} and \texttt{RP}, \texttt{s} knows the public key of \texttt{C},
	and \texttt{RP} knows the public key of \texttt{s}.
	In the cases where the knowledge of public keys are mirrored, that is, the recipient
	knows the public key of the sender, and the other way around, a secure channel can be
	established. This is achieved by the sender encrypting messages in the recipients
	public key, providing confidentiality, and his own private key, providing authentication.
	
	In cases where the recipient does not know the public key of the sender,
	only confidentiality is ensured.

	In this specific case, a secure channel can be established between \texttt{C} and \texttt{s}.
	Additionally, confidential channels can be established between
	\begin{itemize*}
		\item \texttt{C} and \texttt{RP}
		\item \texttt{RP} and \texttt{s}
	\end{itemize*}
	with the sender being the former, and the recipient the latter.
}

\Q{What new security relationship does the protocol try to establish?}
\A{
	The protocol tries to establish a secure connection between \texttt{C}
	and \texttt{RP} on which they can share a secret, \texttt{Data}.
}

\Q{How does the protocol try to achieve this?}
\A{
	asdf
}

\subsection{Analyse and explain the attack}

\subsection{Suggest and verify a fix}

\subsection{Revoke \emph{S'} trustworthy status}

\appendix

\section{Code}

\subsection{PKInit}

\subsubsection{Handout code}
\lstinputlisting
	[
		basicstyle=\footnotesize,
		numbers=left, 
		numberstyle=\tiny
	]
	{../code/handouts/PKINIT.AnB}

\subsubsection{Attack trace}
\label{app:pktrace}
\lstinputlisting
	[
		basicstyle=\footnotesize,
		numbers=left,
		numberstyle=\tiny,
		breaklines=true
	]
	{../code/handouts/PKINIT.log}
	
\subsubsection{Fix verification}
\label{app:pkfix-log}
\lstinputlisting
	[
		basicstyle=\footnotesize,
		numbers=left,
		numberstyle=\tiny,
		breaklines=true
	]
	{../code/written/PKINIT_fix.log}

\subsubsection{Diffie-Hellman variant}
\label{app:pkdh}
\lstinputlisting
	[
		basicstyle=\footnotesize,
		numbers=left,
		numberstyle=\tiny,
		breaklines=true
	]
	{../code/written/PKINIT_dh.AnB}

\subsubsection{Variant verification}
\label{app:pkdh-log}
\lstinputlisting
	[
		basicstyle=\footnotesize,
		numbers=left,
		numberstyle=\tiny,
		breaklines=true
	]
	{../code/written/PKINIT_dh.log}
	
\subsection{AMP}

\subsubsection{Handout code}
\lstinputlisting
	[
		basicstyle=\footnotesize,
		numbers=left,
		numberstyle=\tiny,
		breaklines=true
	]
	{../code/handouts/AMP.AnB}

\subsubsection{Attack trace}
\label{app:amptrace}
\lstinputlisting
	[
		basicstyle=\footnotesize,
		numbers=left,
		numberstyle=\tiny,
		breaklines=true
	]
	{../code/handouts/AMP.log}

\end{document}
